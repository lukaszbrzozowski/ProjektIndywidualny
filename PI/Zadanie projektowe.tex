\documentclass[12pt]{article}

\usepackage[T1]{fontenc}
\usepackage[utf8]{inputenc}
\usepackage{polski}
\usepackage{amssymb, amsfonts, amsmath}
\usepackage{graphicx}
\usepackage{verbatim}
\usepackage{hyperref}
\usepackage{enumerate}
\usepackage{blindtext}
\begin{document}
\title{Projekt aplikacji do analizy i wizualizacji danych wrażeń synestetycznych}
\author{Łukasz Brzozowski}
\maketitle
\begin{figure}[ht!]
	\centering
	\includegraphics[width=\textwidth]{MiNI.png}
\end{figure}
\pagebreak
\tableofcontents
\pagebreak
\section{Wprowadzenie}
\subsection{Opis problemu}
	Celem projektu jest stworzenie aplikacji, która na podstawie danej próbki tekstu będzie dynamicznie tworzyć animacje mające odwzorować wrażenia osób z przypadłością synestezji wzrokowo-leksykalnej. Dzięki takiej aplikacji nastąpi zwiększenie się świadomości społecznej dotyczącej synestezji oraz będzie możliwe przedstawienie codzienności osób z ową cechą. Ponadto, na podstawie danych zebranych od synestetyków zostanie przeprowadzona analiza połączeń kolorystoczno-literowych umożliwiająca odnalezienie najczęstszych wzorców oraz schematów przyporządkowania kolorów. 
\begin{figure}[ht!]
	\centering
	\fbox{\includegraphics[width=\textwidth]{Prototyp_programu.png}}
	\caption{Przykładowy interfejs aplikacji dynamicznie tworzącej animacje}
\end{figure}

\end{document}
